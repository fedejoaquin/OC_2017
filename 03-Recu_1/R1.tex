% ORGANIZACIÓN DE COMPUTADORAS
% SEGUNDO CUATRIMESTRE 2017
% RECUPERATORIO PRIMER EXAMEN PARCIAL

\documentclass[12pt,a4paper]{article}
\input{estilo/Catedras.sty}

\begin{document}

\Examen{Recuperatorio Primer Examen Parcial}

\begin{centering}
\emph{Apague cualquier dispositivo electrónico en su poder y manténgalo guardado. No puede utilizar auriculares, ni calculadora. Lea todo el ejercicio antes de comenzar a desarrollarlo.}	
\end{centering}

\PEjercicio{1} \textbf{¡¡NUEVO!!} Dado el número \textbf{decimal} $-393.3125$ llevar adelante los siguientes cambios de base:
\begin{enumerate}[a)]
	\item Convertirlo a \textbf{octal}, empleando el método de la \textbf{multiplicación} tanto para la parte entera como para la parte fraccionaria, expresando el resultado en \textbf{complemento a la base}, con 4 dígitos octales para la parte entera y 5 dígitos octales para la parte fraccionaria.

	\item Convertirlo a \textbf{hexadecimal} utilizando el método de la \textbf{división} tanto para la parte entera como para la parte fraccionaria, expresando el resultado en \textbf{complemento a la base disminuida}, con 5 digitos hexadecimales tanto para la parte entera como para la parte fraccionaria.
\end{enumerate}

\PEjercicio{2} \textbf{¡¡NUEVO!!} Considerando los números {\textbf{decimales}} $X = 142$ e $Y = 113$, llevar adelante las siguientes operaciones, indicando claramente el resultado obtenido y la existencia o no de \emph{overflow}:
\begin{enumerate}[a)]
	\item Calcular $X + Y$, trabajando en \textbf{binario} en \textbf{complemento a la base}, con una precisión de nueve dígitos (incluido el signo).
	\item Calcular $ - X - Y $, trabajando en \textbf{binario} en \textbf{complemento a la base disminuida}, con una precisión de nueve dígitos (incluido el signo).
	\item Calcular $X - Y$, haciendo uso de un hardware que opera en una codificación \textbf{BCD Exceso-3} y \textbf{complemento a la base disminuida}, con una precisión de cuatro dígitos (incluido el signo), indicando claramente qué operación se está realizando en cada uno de los pasos intermedios.
\end{enumerate}

\PEjercicio{3} \textbf{¡¡NUEVO!!} Considerando el Código Cíclico Redundante (CRC):
\begin{enumerate}[a)]
	\item Construir el mensaje $T(x)$ a transmitir asociado al mensaje de dato $M(x) = 100\,1101\,1001$, empleando el polinomio generador $G(x) = x^4 + x^3 + x^2 + x + 1$.

	\item Suponiendo que durante la transmisión el mensaje $T(x)$ es modificado y el receptor recibe un $T'(x)$ tal que $T'(x) = T(x) \oplus E(x)$, donde $E(x) = x^2 (x^6 + x^2 + 1)$:
	\begin{enumerate}[b.1)]
		\item ¿Cuál es el mensaje $T'(x)$ recibido?
		\item ¿Cuál es la longitud de la ráfaga de error? ¿Es capaz CRC de detectar dicha ráfaga? Justificar.
		\item Mostrar cómo opera el mecanismo de detección de errores y cuál es la conclusión que se alcanza.
	\end{enumerate} 
\end{enumerate}

\PEjercicio{4} Considerando el código Hamming mínima distancia 4 (Hamming extendido), empleando paridad par y estando la secuencia ordenada de izquierda a derecha:
\begin{enumerate}[a)]

	\item Calcular los bits de código asociados al dato $0110\,1011$ y armar el codeword correspondiente que integra el dato y los bits calculados. ¿Cuántos bits de código se tienen que completar? Justifique su respuesta. 

	\item Considerando que el receptor recibe el codeword $1\,0111\,1011\,0110$ que contiene los bits de dato y de código $C_i$. Recalcular los bits de código y determinar cuál es el síndrome.

	\item Determinar cómo trabaja el mecanismo de detección/corrección ante una política $d=2$, $c=1$, con los resultados obtenidos en el inciso b).

	\item Determinar cómo trabaja el mecanismo de detección/corrección ante una política $d=3$, $c=0$, si el síndrome fuera $1110$.

	\item Determinar cómo trabaja el mecanismo de detección/corrección ante una política $d=2$, $c=1$, si el síndrome fuera $0000$.

\end{enumerate}

\PEjercicio{5} \textbf{¡¡NUEVO!!} Dada la definición para \texttt{TCadena}, implementar en \textbf{lenguaje C}:
\begin{enumerate}[a)]
	\item Una función \texttt{int es\_palindroma(TCadena cad)} que dada una cadena de caracteres \texttt{cad}, retorne 1 si \texttt{cad} es \textit{palíndroma}, y cero en caso contrario. Una cadena de caracteres se dice \textit{palíndroma}, si se lee de igual forma de izquierda a derecha que de derecha a izquierda. Ejemplo: ``anana'' y ``neuquen'', son cadenas \textit{palíndromas}, mientras ``parcial'' no lo es.
	\item Una función \texttt{TCadena[] clonar (TCadena[] arr, int long)}, que dado un arreglo de cadenas de caracteres \texttt{arr} de longitud \texttt{long}, retorne un nuevo arreglo de cadenas de igual longitud que \texttt{arr}, pero que contenga la \textit{clonación en profundidad} de aquellas cadenas en \texttt{arr} que son \textit{palíndromas}. Para esto, contemplar el uso de la función definida en el inciso anterior, así como un correcto uso de la función \texttt{malloc} para reservar memoria a la hora de clonar las cadenas.
\end{enumerate}
Dado un \textit{Arreglo A}, la función \texttt{clonar()} retornará un nuevo \textit{Arreglo B}, tal como se indica en la siguiente figura:\\
\begin{centering}
	\includegraphics[height=0.2\textheight, width=1.05\textwidth]{Ejercicio_5.pdf} 
\end{centering}
\end{document}

