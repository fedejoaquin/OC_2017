% ORGANIZACIÓN DE COMPUTADORAS
% SEGUNDO CUATRIMESTRE 2017
% SEGUNDO EXAMEN PARCIAL

\documentclass[12pt,a4paper]{article}
\input{estilo/Catedras.sty}

\begin{document}

\Examen{Segundo Examen Parcial}

\begin{centering}
\emph{Apague cualquier dispositivo electrónico en su poder y manténgalo guardado. No puede utilizar auriculares. Lea todo el ejercicio antes de comenzar a desarrollarlo.}	
\end{centering}

\PEjercicio{1}  En el marco de la norma IEEE 754, considerando la representación en punto flotante de media precisión: mantisa fraccionaria en signo magnitud con hidden bit, exponente en exceso y base 2 y la siguiente distribución de bits:
\begin{center}
	\begin{tabular}{|c|c|c|}\hline
		Sig (1b) & Exponente (8 bits) & Mantisa (10 bits)\\\hline
	\end{tabular}
\end{center}

Dados los números:\quad
$$X = (1\; 10110\; 0011111001) \quad Y = (0\; 00111\; 1000111100)$$


Realizar el producto $X \times Y$ aplicando redondeo por proximidad hacia los pares y hacia $+\infty$, explicando cada uno de los pasos involucrados e indicando claramente qué se hace con los bits G, R y S del resultado y con R y S al redondear. El resultado debe ser expresando según la representación enunciada.



%\ begin{verbatim}
% exponente X = 10010 - 1111 = 11
% exponente Y = 10101 - 1111 = 110
%
%          1.1101001010 x 2^3
% x        1.0010110110 x 2^6
% ------------------------
% 10.001001010|11010011100 x 2^9
%              GR----S=1
%
% Normalizar - No entra G
% 1.0001001010|11010011100 x 2^10
%              R----S=1
%\end{verbatim}


\PEjercicio{2} Hacer

\PEjercicio{3} Determinar cuál es el contenido final de cada uno de los registros y posiciones de memoria involucrados en la siguiente secuencia de instrucciones. Indicar en cada caso, el número de instrucción que origina cada cambio. Asumir que el primer operando es el destino y el segundo la fuente de información para la operación.
\begin{center}
	\begin{minipage}{0.4\textwidth}
		\begin{enumerate}[(1)]
			\itemsep -5pt
			\item mov R1,\#0200
			\item mov (R1), \#0100
			\item mov 0100(R1), R1
			\item mov R2, \#0500
			\item mov @0100(R1), \#0500
			\item mov (0200), 0300
			\item mov R3, 0200
			\item mov R3, @0100(R3)
		\end{enumerate}
	\end{minipage}
	\begin{minipage}{0.4\textwidth}
		Interpretación
		\begin{tabular}{ll}
			\#xxxx   & Inmediato\\
			R       & Registro \\
			(R)     & Registro indirecto\\
			xxxx    & Absoluto \\
			xxxx(R) & Indexado \\
			(xxxx)  & Memoria indirecto\\
			@xxxx(R) & Pre-indexado indirecto
		\end{tabular}
	\end{minipage}
\end{center}

\PEjercicio{4} Hacer

\PEjercicio{5} Considerando el siguiente programa para la arquitectura OCUNS, en la que toda lectura/escritura sobre la dirección FF es redireccionada a la E/S estándar: \\ [2.5mm]
\begin{small}
	\begin{minipage}{.4\textwidth}
		\begin{verbatim}
			      LDA R0, FFh
			      LOAD R1, 0(R0)
			      LOAD R2, 0(R0)
			      XOR R3, R3, R3
			      LDA R4, lbl3
			      JZ R1, lbl3
			      JZ R2, lbl3
			      SUB R5, R1, R2
			      JG R5, lbl2
			lbl1: ADD R3, R3, R2
			      DEC R1
			      JZ R1, lbl3
			      JMP R4
			lbl2: ADD R3, R3, R1
			      DEC R2
			      JZ R2, lbl2
			lbl3: STORE R3, 0(R0)
			      HLT
		\end{verbatim}
	\end{minipage}
	\begin{minipage}{.6\textwidth}	
		\begin{tabular}{|c|l|c|l|}\hline
			\textsc{Op.} & \textsc{Descr.} & \textsc{Form.} & \textsc{Pseudocódigo} \\ \hline
			\textbf{0} & \textsf{add} & \textbf{I} & \texttt{R[d] $\leftarrow$ R[s] + R[t]} \\
			\textbf{1} & \textsf{sub} & \textbf{I} & \texttt{R[d] $\leftarrow$ R[s] - R[t]} \\
			\textbf{2} & \textsf{and} & \textbf{I} & \texttt{R[d] $\leftarrow$ R[s] \& R[t]} \\
			\textbf{3} & \textsf{xor} & \textbf{I} & \texttt{R[d] $\leftarrow$ R[s] \^{} R[t]} \\
			\textbf{4} & \textsf{lsh} & \textbf{I} & \texttt{R[d] $\leftarrow$ R[s] <{}<{} R[t]} \\
			\textbf{5} & \textsf{rsh}  & \textbf{I} & \texttt{R[d] $\leftarrow$ R[s] >{}>{} R[t]} \\
			\textbf{6} & \textsf{load}  & \textbf{I} & \texttt{R[d] $\leftarrow$ mem[offset + R[s]]} \\
			\textbf{7} & \textsf{store} & \textbf{I} & \texttt{mem[offset + R[d]] $\leftarrow$ R[s]} \\
			\textbf{8} & \textsf{lda}   & \textbf{II} & \texttt{R[d] $\leftarrow$ addr} \\
			\textbf{9} & \textsf{jz}    & \textbf{II} & \texttt{if (R[d] == 0) PC $\leftarrow$ PC + addr} \\
			\textbf{A} & \textsf{jg}    & \textbf{II} & \texttt{if (R[d] >{} 0) PC $\leftarrow$ PC + addr} \\
			\textbf{B} & \textsf{call}  & \textbf{II} & \texttt{R[d] $\leftarrow$ PC; PC $\leftarrow$ addr} \\
			\textbf{C} & \textsf{jmp}   & \textbf{III} & \texttt{PC $\leftarrow$ R[d]} \\
			\textbf{D} & \textsf{inc}   & \textbf{III} & \texttt{R[d] $\leftarrow$ R[d] + 1} \\
			\textbf{E} & \textsf{dec}   & \textbf{III} & \texttt{R[d] $\leftarrow$ R[d] - 1} \\
			\textbf{F} & \textsf{hlt}   & \textbf{III} & \texttt{exit} \\ \hline
		\end{tabular}	
	\end{minipage}
	\begin{center}
		\begin{tabular}{*{17}{c}}
			\textsf{FORMATO} & 15 & 14 & 13 & 12 & 11 & 10 & 9 & 8 & 7 & 6 & 5 & 4 & 3 & 2 & 1 & 0 \\ \hline
			\multicolumn{1}{|c|}{\textbf{I}} & 0 & $\times$ & $\times$ & $\times$ &
			\multicolumn{4}{|c|}{\textsf{dest. d}} &
			\multicolumn{4}{|c|}{\textsf{src. s}} &
			\multicolumn{4}{|c|}{\textsf{src. t / off.}} \\ \hline
			\multicolumn{1}{|c|}{\textbf{II}} & 1 & 0 & $\times$ & $\times$ &
			\multicolumn{4}{|c|}{\textsf{dest. d}} &
			\multicolumn{8}{|c|}{\textsf{address addr}} \\ \hline
			\multicolumn{1}{|c|}{\textbf{III}} & 1 & 1 & $\times$ & $\times$ &
			\multicolumn{4}{|c|}{\textsf{dest. d}} &
			\multicolumn{8}{|c|}{\textsf{-}} \\ \hline
		\end{tabular}
	\end{center}		
\end{small}

\begin{enumerate}[a)]
	\item Ensamblar el programa a partir de la dirección 00h. \label{ensamblado}
	\item Si se reubicara el código máquina obtenido en el inciso (\ref{ensamblado}) a partir de la dirección 20h, ¿qué referencias a memoria requieren ser ajustadas? Justificar adecuadamente.
	\item Suponiendo que los valores ingresados por teclado son 1Ah y 04h:
	\begin{enumerate}[c.1)]
		\item Realice una traza mostrando la evolución del contenido de cada registro.
		\item Describir el propósito del programa en su conjunto.
		\item ¿Qué sucede con el resultado retornado por el programa si los valores ingresados fueran 04h y 1Ah? ¿Cuál es la diferencia?
		\item ¿Cuál es la restricción en cuanto a los valores de entrada que considera el programa para su correcto funcionamiento? 
	\end{enumerate}
\end{enumerate}

\end{document}


