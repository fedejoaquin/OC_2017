% ORGANIZACIÓN DE COMPUTADORAS
% SEGUNDO CUATRIMESTRE 2017
% RECUPERATORIO SEGUNDO EXAMEN PARCIAL

\documentclass[12pt,a4paper]{article}
\input{estilo/Catedras.sty}

\begin{document}

\Examen{Recuperatorio Segundo Examen Parcial}

\PEjercicio{1} 
Implementar la siguiente expresión aritmética $B = (((D + C)^2 )  + (A \times (D + C))\times (D + C))$, siendo $A, B, C$ y $D$ etiquetas que denotan \textit{direcciones de memoria}, y asumiendo que se cuenta con las instrucciones \texttt{add} y \texttt{mpy}, para las siguientes arquitecturas:
\begin{enumerate}[a)]
	\item Una arquitectura de \textbf{0--direcciones} (tipo pila), contando con la instrucción \texttt{dup} (duplica el tope de la pila). Determinar la profundidad de la pila alcanzada. 
	
	\item Una arquitectura estilo \textbf{RISC}, registro a registro, sin restricción en la cantidad de registros, y con instrucciones \texttt{lda}, \texttt{ld} y \texttt{st}. Indicar la cantidad de accesos a memoria realizados.
	
	\item Una arquitectura de \textbf{1--dirección + registro} (tipo Intel), sin restricción en la cantidad de registros y con la instrucción \texttt{mov}. Indicar la cantidad de accesos a memoria requeridos.	
\end{enumerate}

\PEjercicio{2} En el marco de la norma \textbf{IEEE 754}, considerando la representación en punto flotante simplificada: mantisa fraccionaria en signo magnitud con hidden bit, exponente en exceso y base 2 y la siguiente distribución de bits:
\begin{center}
	\begin{tabular}{|c|c|c|}\hline
		Sig (1bit) & Exponente (8 bits) & Mantisa (10 bits)\\\hline
	\end{tabular}
\end{center}

Dados los números $X = (0\; 01111101\; 0000000111)$ e $Y = (0\; 01111010 \; 0000001001)$ realizar el producto $X \times Y$ aplicando redondeo por proximidad hacia los pares, explicando cada uno de los pasos involucrados e indicando claramente qué se hace con los bits \textbf{G}, \textbf{R} y \textbf{S} del resultado y con \textbf{R} y \textbf{S} al redondear. El resultado debe ser expresando según la representación enunciada. Finalmente, convierta el número hallado a decimal e indique el error existente entre este valor y el obtenido al operar la multiplicación directamente sobre $X$ e $Y$ en decimal.

\textit{(Pista: $X=0.251708984$, $Y= 0.031524658$, $X \times Y = 0,00793503969907761$)}.

\PEjercicio{3} Considerando la representación en punto flotante propuesta para el ejercicio anterior, y los números $X = (0 \ 01111100 \ 0010000000)$ e $Y = (0 \ 01111001 \ 0000110011)$, realizar la suma $X + Y$ aplicando redondeo hacia $+\infty$, explicando cada uno de los pasos involucrados e indicando claramente qué se hace con los bits \textbf{G}, \textbf{R} y \textbf{S} del resultado y con \textbf{R} y \textbf{S} al redondear. El resultado debe ser expresando según la representación enunciada. Finalmente, convierta el número hallado a decimal e indique el error existente entre este valor y el obtenido al operar la suma directamente sobre $X$ e $Y$ en decimal. 

\textit{(Pista: $X=0.140625$, $Y=0,016403198$, $X+Y = 0,157028198242187$)}.

\newpage

\PEjercicio{4} Dados los valores indicados tanto para el banco de registros como para las etiquetas de memoria, indicar para cada modo de direccionamiento, el registro \texttt{R} y/o el número hexadecimal \texttt{xxxx} necesarios para mover el operando \texttt{100h} al registro \texttt{R6}. Luego, indicar en cada paso cuántos accesos a memoria se realizan por la instrucción.
\begin{center}
	\begin{minipage}{0.17\textwidth}
		\begin{tabular}{|c|c|} \hline
			Reg. & Cont. \\\hline
			R1 & 100h \\\hline
			R2 & 200h \\\hline
			R3 & 300h \\\hline
			R4 & 400h \\\hline
		\end{tabular}
	\end{minipage}
	\begin{minipage}{0.17\textwidth}
		\begin{tabular}{|c|c|} \hline
			Dir. & Cont. \\\hline
			100h & 500h \\\hline
			200h & 400h \\\hline
			400h & 100h \\\hline
			600h & 400h \\\hline
		\end{tabular}
	\end{minipage}
	\begin{minipage}{0.3\textwidth}
		\begin{enumerate}[(1)]
			\itemsep -5pt
			\item mov R6, \#xxxx
			\item mov R6, R
			\item mov R6, (R)
			\item mov R6, xxxx
			\item mov R6, (xxxx)
			\item mov R6, (R2)xxxx
			\item mov R6, @300(R)
		\end{enumerate}
	\end{minipage}
	\begin{minipage}{0.3\textwidth}
		Interpretación
		\begin{tabular}{ll}
			\#xxxx   & Inmediato\\
			R       & Registro \\
			(R)     & Registro indirecto\\
			xxxx    & Absoluto \\
			(xxxx)  & Memoria indirecto\\
			(R)xxxx & Base \\
			@xxxx(R) & Pre-indexado indirecto
		\end{tabular}
	\end{minipage}
\end{center}

\PEjercicio{5} Considerando el programa A para la arquitectura OCUNS, en la que toda lectura/escritura sobre la dirección FF es redireccionada a la E/S estándar, y los pseudocódigos 1 y 2 indicados a continuación: \\ [2.5mm]
\begin{small}
	\begin{minipage}{.4\textwidth}
		\begin{verbatim}
		Programa A:
			      LDA R0, FFh
			      LOAD R1, 0(R0)
			      ADD R2, RF, RF
			      JZ R1, lbl2
			lbl1: ADD R2, R2, R1
			      DEC R1
			      JG R1, lbl1
			lbl2: STORE R2, 0(R0)
			      HLT
		\end{verbatim}
		\begin{verbatim}
		Pseudocódigo 1
		  if (R1 <= 4) R2++;
		  else R2--;
		\end{verbatim}
		\begin{verbatim}
		Pseudocódigo 2
		  R3 = 0;
		  for(R4 = 0; R4 < 10; R4++)
		      R3 += R4;
		\end{verbatim}
	\end{minipage}
	\begin{minipage}{.6\textwidth}	
		\begin{tabular}{|c|l|c|l|}\hline
			\textsc{Op.} & \textsc{Descr.} & \textsc{Form.} & \textsc{Pseudocódigo} \\ \hline
			\textbf{0} & \textsf{add} & \textbf{I} & \texttt{R[d] $\leftarrow$ R[s] + R[t]} \\
			\textbf{1} & \textsf{sub} & \textbf{I} & \texttt{R[d] $\leftarrow$ R[s] - R[t]} \\
			\textbf{2} & \textsf{and} & \textbf{I} & \texttt{R[d] $\leftarrow$ R[s] \& R[t]} \\
			\textbf{3} & \textsf{xor} & \textbf{I} & \texttt{R[d] $\leftarrow$ R[s] \^{} R[t]} \\
			\textbf{4} & \textsf{lsh} & \textbf{I} & \texttt{R[d] $\leftarrow$ R[s] <{}<{} R[t]} \\
			\textbf{5} & \textsf{rsh}  & \textbf{I} & \texttt{R[d] $\leftarrow$ R[s] >{}>{} R[t]} \\
			\textbf{6} & \textsf{load}  & \textbf{I} & \texttt{R[d] $\leftarrow$ mem[offset + R[s]]} \\
			\textbf{7} & \textsf{store} & \textbf{I} & \texttt{mem[offset + R[d]] $\leftarrow$ R[s]} \\
			\textbf{8} & \textsf{lda}   & \textbf{II} & \texttt{R[d] $\leftarrow$ addr} \\
			\textbf{9} & \textsf{jz}    & \textbf{II} & \texttt{if (R[d] == 0) PC $\leftarrow$ PC + addr} \\
			\textbf{A} & \textsf{jg}    & \textbf{II} & \texttt{if (R[d] >{} 0) PC $\leftarrow$ PC + addr} \\
			\textbf{B} & \textsf{call}  & \textbf{II} & \texttt{R[d] $\leftarrow$ PC; PC $\leftarrow$ addr} \\
			\textbf{C} & \textsf{jmp}   & \textbf{III} & \texttt{PC $\leftarrow$ R[d]} \\
			\textbf{D} & \textsf{inc}   & \textbf{III} & \texttt{R[d] $\leftarrow$ R[d] + 1} \\
			\textbf{E} & \textsf{dec}   & \textbf{III} & \texttt{R[d] $\leftarrow$ R[d] - 1} \\
			\textbf{F} & \textsf{hlt}   & \textbf{III} & \texttt{exit} \\ \hline
		\end{tabular}	
	\end{minipage}
	\begin{center}
		\begin{tabular}{*{17}{c}}
			\textsf{FORMATO} & 15 & 14 & 13 & 12 & 11 & 10 & 9 & 8 & 7 & 6 & 5 & 4 & 3 & 2 & 1 & 0 \\ \hline
			\multicolumn{1}{|c|}{\textbf{I}} & 0 & $\times$ & $\times$ & $\times$ &
			\multicolumn{4}{|c|}{\textsf{dest. d}} &
			\multicolumn{4}{|c|}{\textsf{src. s}} &
			\multicolumn{4}{|c|}{\textsf{src. t / off.}} \\ \hline
			\multicolumn{1}{|c|}{\textbf{II}} & 1 & 0 & $\times$ & $\times$ &
			\multicolumn{4}{|c|}{\textsf{dest. d}} &
			\multicolumn{8}{|c|}{\textsf{address addr}} \\ \hline
			\multicolumn{1}{|c|}{\textbf{III}} & 1 & 1 & $\times$ & $\times$ &
			\multicolumn{4}{|c|}{\textsf{dest. d}} &
			\multicolumn{8}{|c|}{\textsf{-}} \\ \hline
		\end{tabular}
	\end{center}		
\end{small}
\begin{small}
\begin{enumerate}[a)]
	\item Ensamblar el programa A a partir de la dirección 00h.
	\item Suponiendo que se ingresa por teclado el valor 03h, realice una traza del programa A mostrando la evolución del contenido de cada registro y del PC (paso a paso), y luego describa el propósito del programa en su conjunto.
	\item Indique una secuencia de instrucciones para la arquitectura OCUNS, que sea equivalente al pseudocódigo 1.
	\item Indique una secuencia de instrucciones para la arquitectura OCUNS, que sea equivalente al pseudocódigo 2.	
\end{enumerate}
\end{small}
\end{document}


