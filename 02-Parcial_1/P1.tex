% ORGANIZACIÓN DE COMPUTADORAS
% SEGUNDO CUATRIMESTRE 2017
% PRIMER EXAMEN PARCIAL

\documentclass[12pt,a4paper]{article}
\input{estilo/Catedras.sty}

\begin{document}

\Examen{Primer Examen Parcial}

\begin{center}
	\emph{Apague cualquier dispositivo electrónico en su poder y manténgalo guardado. No puede utilizar auriculares, ni calculadora. Lea todo el ejercicio antes de comenzar a desarrollarlo.}
\end{center}

\PEjercicio{1}Dado el número \textbf{decimal} $-298.5625$ llevar adelante los siguientes cambios de base:
\begin{enumerate}[a)]
	\item Convertirlo a \textbf{octal}, empleando el método de la \textbf{división} tanto para la parte entera como para la parte fraccionaria, expresando el resultado en \textbf{complemento a la base}, con 4 dígitos octales para la parte entera y 6 para la parte fraccionaria.

	\item Convertirlo a \textbf{binario} utilizando el método de la \textbf{multiplicación} tanto para la parte entera como para la parte fraccionaria, expresando el resultado en \textbf{complemento a la base disminuida}, con 12 bits para la parte entera y 6 bits para la parte fraccionaria.
\end{enumerate}

\PEjercicio{2}Considerando los números {\textbf{decimales}} $X = 1537$ e $Y = 2559$, llevar adelante las siguientes operaciones con una precisión de cuatro dígitos (incluido el signo), indicando claramente el resultado obtenido y la existencia o no de \emph{overflow}:
\begin{enumerate}[a)]
	\item Calcular $- X - Y$, trabajando en \textbf{hexadecimal} en \textbf{complemento a la base}.
	\item Calcular $ X + Y $, trabajando en \textbf{hexadecimal} en \textbf{complemento a la base disminuida}.
	\item Calcular $X - Y$, haciendo uso de un hardware que opera en una codificación \textbf{BCD Exceso-3} y \textbf{complemento a la base}, indicando claramente qué operación se está realizando en cada uno de los pasos intermedios.
\end{enumerate}

\PEjercicio{3} Considerando el Código Cíclico Redundante (CRC):

\begin{enumerate}[a)]
	\item Construir el mensaje $T(x)$ a transmitir asociado al mensaje de datos
	$M(x) = 110\,1011\,1011$ empleando el polinomio generador $G(x) = x^3 + x^2 + 1$.

	\item Suponiendo que durante la transmisión el mensaje $T(x)$ es modificado con un error $E(x)$ de tal forma que el receptor recibe el mensaje $T'(x) = 101\,1001\,0010\,1001$, determinar cómo opera el mecanismo de detección de errores y cuál es la conclusión que se alcanza.

	\item Comparando el mensaje transmitido $T(x)$ y el mensaje recibido $T'(x)$, ¿cuál es el desarrollo del polinomio de error $E(x)$? Sabiendo cuál fue el error exacto que existió, ¿cuál es la longitud de la ráfaga en error? y ¿a qué conclusión se puede arribar?
\end{enumerate}


\PEjercicio{4} Considerando el código Hamming mínima distancia 3 (Hamming básico), empleando paridad par y estando la secuencia ordenada de izquierda a derecha:

\begin{enumerate}[a)]

	\item Calcular los bits de código asociados al dato $0110\,1011$ y armar el codeword correspondiente que integra el dato y los bits calculados. 

	\item Considerando que el receptor recibe el codeword $1100\,0101\,1010$ que contiene los bits de dato y de código $C_i$. Recalcular los bits de código y determinar cuál es el síndrome.

	\item Determinar cómo trabaja el mecanismo de detección/corrección antes una política $d=2$, $c=1$ cuando al recalcular los bits de código de un codeword recibido se obtiene síndrome $0100$.

	\item Determinar cómo trabaja el mecanismo de detección/corrección antes una política $d=3$, $c=0$ cuando al recalcular los bits de código de un codeword recibido se obtiene síndrome $1010$.
\end{enumerate}

\PEjercicio{5} Dadas las siguientes declaraciones de tipos:
\begin{verbatim}
	typedef void * tElemento;
	typedef struct nodo{
	   tElemento elemento;
	   struct nodo * padre;
	   struct nodo * hijo_izq;
	   struct nodo * hijo_der;
	} * tNodo;
	typedef struct abb{
	   unsigned int cant_elementos;
	   struct nodo * raiz;
	} * tABB;
\end{verbatim}
Implementar en lenguaje C, ACÁ DEBERÍAMOS SELECCIONAR ALGUNA DE LAS SIGUIENTES OPCIONES y la otra la podemos dejar para el RECU :) 
\begin{itemize}
	\item una función \texttt{void insertar(tABB a, tElemento e, int (*f)(void *,void *))}, que dado un \emph{Árbol Binario de Búsqueda} y una función de comparación \texttt{f}, inserte el elemento \texttt{e} en su correspondiente ubicación, haciendo un correcto uso de la memoria dinámica. Se considera que la función \texttt{f}, devuelve -1 si el orden del primer argumento es menor que el orden del segundo, 0 si el orden es el mismo, y 1 si el orden del primer argumento es mayor que el orden del segundo. \textbf{Deben hacer uso de malloc al crear un nuevo nodo, y moverse sobre la ED haciendo uso del comparador.}
\end{itemize}

\begin{itemize}
	\item una función \texttt{void eliminar\_raiz(tABB a, tElemento e, int (*f)(void *,void *))}, que dado un \emph{Árbol Binario de Búsqueda}, elimine el elemento almacenado en el nodo raíz del árbol, y reacomode la estructura tABB de forma tal que siga representando un árbol binario de búsqueda. \textbf{Deben hacer uso de free al eliminar un nodo hoja que pasa a ser nueva raiz, y moverse sobre la ED.}
\end{itemize}  

\end{document}

