% ORGANIZACIÓN DE COMPUTADORAS
% SEGUNDO CUATRIMESTRE 2017
% PRIMER EXAMEN PARCIAL

\documentclass[12pt,a4paper]{article}
\input{estilo/Catedras.sty}

\begin{document}

\Examen{Primer Examen Parcial}

\begin{centering}
\emph{Apague cualquier dispositivo electrónico en su poder y manténgalo guardado. No puede utilizar auriculares, ni calculadora. Lea todo el ejercicio antes de comenzar a desarrollarlo.}	
\end{centering}

\PEjercicio{1}Dado el número \textbf{decimal} $-298.5625$ llevar adelante los siguientes cambios de base:
\begin{enumerate}[a)]
	\item Convertirlo a \textbf{octal}, empleando el método de la \textbf{división} tanto para la parte entera como para la parte fraccionaria, expresando el resultado en \textbf{complemento a la base}, con 4 dígitos octales para la parte entera y 6 para la parte fraccionaria.

	\item Convertirlo a \textbf{binario} utilizando el método de la \textbf{multiplicación} tanto para la parte entera como para la parte fraccionaria, expresando el resultado en \textbf{complemento a la base disminuida}, con 12 bits para la parte entera y 6 bits para la parte fraccionaria.
\end{enumerate}

\PEjercicio{2}Considerando los números {\textbf{decimales}} $X = 1537$ e $Y = 2559$, llevar adelante las siguientes operaciones, indicando claramente el resultado obtenido y la existencia o no de \emph{overflow}:
\begin{enumerate}[a)]
	\item Calcular $- X - Y$, trabajando en \textbf{hexadecimal} en \textbf{complemento a la base}, con una precisión de cuatro dígitos (incluido el signo).
	\item Calcular $ X + Y $, trabajando en \textbf{hexadecimal} en \textbf{complemento a la base disminuida}, con una precisión de cuatro dígitos (incluido el signo).
	\item Calcular $X - Y$, haciendo uso de un hardware que opera en una codificación \textbf{BCD Exceso-3} y \textbf{complemento a la base}, con una precisión de cinco dígitos (incluido el signo), indicando claramente qué operación se está realizando en cada uno de los pasos intermedios.
\end{enumerate}

\PEjercicio{3} Considerando el Código Cíclico Redundante (CRC):
\begin{enumerate}[a)]
	\item Construir el mensaje $T(x)$ a transmitir asociado al mensaje de datos
	$M(x) = 110\,1011\,1011$ empleando el polinomio generador $G(x) = x^4 + x + 1$.

	\item Suponiendo que durante la transmisión el mensaje $T(x)$ es modificado con un error $E(x)$ de tal forma que el receptor recibe el mensaje $T'(x) = 110\,\, 0011\,0011\,1011$, determinar cómo opera el mecanismo de detección de errores y cuál es la conclusión que se alcanza.

	\item Comparando el mensaje transmitido $T(x)$ y el mensaje recibido $T'(x)$, ¿cuál es el desarrollo del polinomio de error $E(x)$? Sabiendo cuál fue el error exacto que existió: ¿cuál es la longitud de la ráfaga en error? ¿A qué conclusión se puede arribar?
\end{enumerate}

\PEjercicio{4} Considerando el código Hamming mínima distancia 4 (Hamming extendido), empleando paridad par y estando la secuencia ordenada de izquierda a derecha:
\begin{enumerate}[a)]

	\item Calcular los bits de código asociados al dato $0110\,1011$ y armar el codeword correspondiente que integra el dato y los bits calculados. ¿Cuántos bits de código se tienen que completar? Justifique su respuesta. 

	\item Considerando que el receptor recibe el codeword $1\,0111\,1011\,0110$ que contiene los bits de dato y de código $C_i$. Recalcular los bits de código y determinar cuál es el síndrome.

	\item Determinar cómo trabaja el mecanismo de detección/corrección ante una política $d=2$, $c=1$, con los resultados obtenidos en el inciso b).

	\item Determinar cómo trabaja el mecanismo de detección/corrección ante una política $d=3$, $c=0$, si el síndrome fuera $1110$.

	\item Determinar cómo trabaja el mecanismo de detección/corrección ante una política $d=2$, $c=1$, si el síndrome fuera $0000$.

\end{enumerate}

\PEjercicio{5} Dados los tipos \texttt{tElemento}, \texttt{tNodo} y \texttt{tABB}, implementar en \textbf{lenguaje C}: 
\begin{itemize}
	\item Un procedimiento \texttt{void insertar(tABB a, tElemento e, int (*f)(void *,void *))}, que dado un \textit{árbol binario de búsqueda} \texttt{a} y una \textit{función de comparación} \texttt{f}, inserte el elemento \texttt{e} en su correspondiente ubicación, haciendo un correcto uso de la memoria dinámica. 
\end{itemize}

Se considera que la función \texttt{f}, retorna -1 si el orden del primer argumento es menor que el orden del segundo, 0 si el orden es el mismo, y 1 si el orden del primer argumento es mayor que el orden del segundo. Dado un árbol para elementos enteros inicialmente vacío, la inserción de los elementos \textit{5, 3, 6, 4}, se debe resolver tal como se indica a continuación: \\
\begin{center}
	\includegraphics[width=1\textwidth]{Ejercicio_5.pdf}
	\textit{Figura 1: \texttt{tElemento}, \texttt{tNodo}, \texttt{tABB}, y gráfica de inserción de los elementos 5, 3, 6 y 4, comenzando desde un árbol vacío. Por simplificación, se omitieron las referencias al padre de cada nodo.}
\end{center}
\end{document}


