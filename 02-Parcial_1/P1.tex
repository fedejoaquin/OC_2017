% ORGANIZACIÓN DE COMPUTADORAS
% SEGUNDO CUATRIMESTRE 2017
% PRIMER EXAMEN PARCIAL

\documentclass[12pt,a4paper]{article}
\input{estilo/Catedras.sty}

\begin{document}

\Examen{Primer Examen Parcial}

\begin{center}
	\emph{Apague cualquier dispositivo electrónico en su poder y manténgalo guardado. No puede utilizar auriculares, ni calculadora. Lea todo el ejercicio antes de comenzar a desarrollarlo.}
\end{center}

\PEjercicio{1}Dado el número \textbf{decimal} $-298.5625$ llevar adelante los siguientes cambios de base:
\begin{enumerate}[a)]
	\item Convertirlo a \textbf{octal}, empleando el método de la \textbf{división} tanto para la parte entera como para la parte fraccionaria, expresando el resultado en \textbf{complemento a la base}, con 4 dígitos decimales para la parte entera y 6 para la parte fraccionaria.

	\item Convertirlo a \textbf{binario} utilizando el método de la \textbf{multiplicación} tanto para la parte entera como para la parte fraccionaria, expresando el resultado en \textbf{complemento a la base disminuida}, con 12 bits para la parte entera y 6 bits para la parte fraccionaria.
\end{enumerate}

\PEjercicio{2}Considerando los números {\textbf{decimales}} $X = 1537$ e $Y = 2559$, llevar adelante las siguientes operaciones con una precisión de cuatro dígitos (incluido el signo), indicando claramente el resultado obtenido y la existencia o no de \emph{overflow}:
\begin{enumerate}[a)]
	\item Calcular $- X - Y$, trabajando en \textbf{hexadecimal} en \textbf{complemento a la base}.
	\item Calcular $ X + Y $, trabajando en \textbf{hexadecimal} en \textbf{complemento a la base disminuida}.
	\item Calcular $X - Y$, haciendo uso de un hardware que opera en una codificación \textbf{BCD Exceso-3} y \textbf{complemento a la base}, indicando claramente qué operación se está realizando en cada uno de los pasos intermedios.
\end{enumerate}

\end{document}

